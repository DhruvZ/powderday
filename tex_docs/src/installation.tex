\section{Overview}

We are working on an all-in-one installer script for \pd \ that will
work on a variety of platforms.  For the time being, for better or
worse, the installation is manual.  The fundamental packages you will
need to have installed are:
\begin{enumerate}
\item \yt 
\item Python Dependencies (listed below)
\item \hyperion
\item \fsps
\item \pfsps
\end{enumerate}

Here, we'll detail the rough order of operations, though it's always
wise to check the parent website (referenced in each subsection) for
details.

\subsection{\yt \ installation}
The parent website is \url{http://yt-project.org}.  There, you will
find at \url{http://yt-project.org/\#getyt} a variety of ways to
install \yt.  We recommend the Development branch, though \pd \ is
currently known to work with the Stable branch.  Note - the Legacy
branch of \yt \ is incompatible with \pd.  You can either install from
the install scripts, or use conda or pip. Note that to use \yt, you
will either need to activate \yt \ via:
\begin{verbatim}
 > source YT\_DEST/bin/activate
\end{verbatim}
or, in your .bashrc:\\
\begin{verbatim}
export PATH = \$HOME/yt-x86_64/bin:\$PATH
\end{verbatim}
By doing this, the python you use (and install packages to,
subsequently) will be the \yt \ python.  If you don't want this to
happen, just ensure that the yt packages are in your python path, and
install the forthcoming packages to whichever python distribution you
would like.



\subsection{Python Dependencies}

\pd \ depends on both \scipy, \astropy, and {\sc ATpy}.  The easiest way to
install these is with pip, following:
\begin{verbatim}
 > pip install scipy 
 > pip install astropy
 > pip install atpy
\end{verbatim}
You can also install them straight from source, which you can get from
\url{http://scipy.org}, and \url{http://astropy.org}.

\subsection{\hyperion \ Installation}
The \hyperion \ installation is described in full on the \hyperion
\ website, and therefore we direct you there:
\url{http://docs.hyperion-rt.org/en/stable/installation/installation.html}
for this.  Note, Tom Robitaille's general host website for \hyperion \ is at:
\url{http://www.hyperion-rt.org}. A few notes:
\begin{itemize}
\item First install the Fortran code dependencies: \url{http://docs.hyperion-rt.org/en/stable/installation/fortran_dependencies.html}
\item Second, install any remaining python dependencies:
  \url{http://docs.hyperion-rt.org/en/stable/installation/python_dependencies.html}.
  Note, if you're using the python that comes with \yt, then you
  should already have all of these installed.
\item Finally, install \hyperion \ itself: \url{http://docs.hyperion-rt.org/en/stable/installation/installation.html#hyperion}
\item Make sure that all of the tests that the \hyperion \ installation page suggests actually work (i.e. when you type:
\begin{verbatim}
>hyperion\_sph
\end{verbatim}
you get something like:
\begin{verbatim}
Usage: hyperion input\_file output\_file
\end{verbatim}
\end{itemize}

\subsection{\hyperion \ Dust Files}
Unless you've written your own dust files (a tutorial for which will
come later in this manual), you might want to use the pre-compiled
dust files that are developed by Tom Robitaille (though don't ship
with \hyperion).  To install these, download and install from
\url{http://docs.hyperion-rt.org/en/stable/dust/dust.html}.

In order to run with the PAH model turned on, you'll additionally need the MW PAH dust files.  There are three (big, very small grains and ultra small grains).  These can be donwloaded here:
\url{https://github.com/hyperion-rt/paper-galaxy-rt-model/blob/master/dust/big.hdf5}\\
\url{https://github.com/hyperion-rt/paper-galaxy-rt-model/blob/master/dust/vsg.hdf5}\\
\url{https://github.com/hyperion-rt/paper-galaxy-rt-model/blob/master/dust/usg.hdf5}\\

\subsection{\fsps}
\fsps \ is hosted at google code, and can be checked out with:
\begin{verbatim}
> svn checkout http://fsps.googlecode.com/svn/trunk/ fsps
\end{verbatim}
and directions to the installation are in the Manual:
\url{https://www.cfa.harvard.edu/~cconroy/FSPS_files/MANUAL.pdf}.
Note, Charlie Conroy's host website for \fsps \ is at:
\url{https://www.cfa.harvard.edu/~cconroy/FSPS.html}.

While \pd \ (more accurately, it's dependencies on \pfsps) may
work with the most current version of \fsps, they are only known to
work for \fsps \ revisison 143.  Therefore, it is safest to revert to
this revision with:
\begin{verbatim}
>svn update -r 143
\end{verbatim}
Note that the SPS\_HOME environment variable must be set in your path
to point to the \fsps/src directory.  For example, in my .bashrc, I have:
\begin{verbatim}
>export SPS\_HOME=/Users/dnarayanan/fsps/
\end{verbatim}
\subsection{\pfsps}
\pd \ depends on python hooks for \fsps \ written by Dan
Foreman-Mackey and others, hosted at:
\url{http://dan.iel.fm/python-fsps/current/installation/}.  
For \pfsps, there is a pip installer that will allow you to install via:
\begin{verbatim}
>pip install fsps
\end{verbatim}
Though you cuold also install the development version with:
\begin{verbatim}
git clone https://github.com/dfm/python-fsps.git
cd python-fsps
python setup.py install
\end{verbatim}

\subsection{TroubleShooting}

So far, the aforementioned has been known to work for Mac OS X, Ubuntu
and CentOS linux operating systems. Of course, troubles are potential.
Here are a few issues that we've come across with possible solutions.

\subsubsection{\fsps \ and \fsps}

\color{red}{\bf Problem}: \color{black} \pfsps \ has an error related to 'fPIC' when installing.\\
\color{blue} {\bf Solution}: \color{black} try include that flag in the FSPS makefile. You can do this by adding -fPIC to the F90FLAGS in the Makefile for FSPS, i.e.
\begin{verbatim}
>F90FLAGS = -O -cpp -fPIC
\end{verbatim}
in your Makefile.\\
\newline
\color{red} {\bf Problem} \color{black} \pfsps \ gives errors related to f2py when installing \\
\color{blue} {\bf Solution} \color{black} To install python-fsps, you
need to have f2py installed. If you're using the \yt \ python, this comes with the \yt \ installation, but
is named f2py2.7 and is in (e.g.)
\begin{verbatim}
\Users/desika/yt-x86_64/bin
\end{verbatim}
So, simply go to that directory and:
\begin{verbatim}
> ln -s f2py2.7 f2py
\end{verbatim}
You additionally need to go to: 
\begin{verbatim}
>cd  yt-x86_64/lib/python2.7/site-package
> ln -s numpy/f2py/ f2py
\end{verbatim}


\subsection{\hyperion}

\color{red}{\bf Problem}: \color{black} \hyperion \ gives an error like:
\begin{verbatim}
ld: library not found for -lcrt1.10.5.o
\end{verbatim}
when compiling (particularly with Mac OS X)\\
\color{blue} {\bf Solution}: \color{black} It could be that there are issues with the command line tools installation (which has cropped up especially for the Mavericks OS).  Try:
\begin{verbatim}
>sudo xcode-select --install
\end{verbatim}
(assuming you have xcode installed already).  This will downlad and
install the command line tools.

\subsection{\pd}
\color{red}{\bf Problem}: \color{black} Memory Errors at the beginning of the Peeled Images setup\\
\color{blue} {\bf Solution}: \color{black} Change the track\_origin from detailed to basic in pd\_front\_end.py

